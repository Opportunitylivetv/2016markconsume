This approach, suggested off-list by JF Bastien, creates a
\co{depending_ptr}\footnote{
	Arbitrarily chosen name with no Google hits.}
template to which a pointer-like type is passed.
This approach allows implementers considerable freedom, as they can
hook into the \co{->} and \co{*} if need be, and also use the C++
\co{delete} keyword to prohibit problematic operations.
Implementations that might nevertheless carry out aggressive
optimizations that might break dependencies even for the non-problematic
operations might need to implement this template class in a manner
similar to the atomics template classes.

This approach would need to be augmented with a non-template solution
for C, for example, the object-modifier approach from
Section~\ref{sec:Object Modifier}.
Implementations that support both C and C++ would presumably relate
Section~\ref{sec:Object Modifier}'s
keyword to the templates in this section in a manner similar to
that used for atomics.

\begin{figure}[tbp]
{ \scriptsize
\begin{verbatim}
 1 template<typename T> class depending_ptr {
 2 public:
 3   depending_ptr(T *v);
 4   depending_ptr();
 5   bool operator!();
 6   class depending_ptr<T>* operator&();
 7   T operator*();
 8   T *operator->();
 9   class depending_ptr<T> operator++();
10   class depending_ptr<T> operator++(int);
11   class depending_ptr<T> operator--();
12   class depending_ptr<T> operator--(int);
13   operator T*();
14   T operator[](long int);
15   bool operator==(T *v);
16   bool operator!=(T *v);
17   bool operator>(T *v);
18   bool operator>=(T *v);
19   bool operator<(T *v);
20   bool operator<=(T *v);
21   class depending_ptr<T> operator+(long idx);
22   class depending_ptr<T> operator+=(long idx);
23   class depending_ptr<T> operator-(long idx);
24   class depending_ptr<T> operator-=(long idx);
25   int operator~() = delete;
26   int operator+() = delete;
27   int operator-() = delete;
28   int operator&(long int) = delete;
29   int operator&=(long int) = delete;
30   int operator%(long int) = delete;
31   int operator%=(long int) = delete;
32   int operator*(long int) = delete;
33   int operator*=(long int) = delete;
34   int operator/(long int) = delete;
35   int operator/=(long int) = delete;
36   int operator<<(long int) = delete;
37   int operator<<=(long int) = delete;
38   int operator>>(long int) = delete;
39   int operator>>=(long int) = delete;
40   int operator^(long int) = delete;
41   int operator^=(long int) = delete;
42   int operator|(long int) = delete;
43   int operator|=(long int) = delete;
44 private:
45   T *dp_rep;
46 };
\end{verbatim}
}
\caption{Template: Declaration}
\label{fig:Template: Declaration}
\end{figure}

Figure~\ref{fig:Template: Declaration}
shows the resulting template declaration, each member function of which
has a straightforward definition.
Note especially that the relational operators are defined in terms of
\co{pointer_cmp_eq_dep()},
\co{pointer_cmp_ne_dep()},
\co{pointer_cmp_gt_dep()},
\co{pointer_cmp_ge_dep()},
\co{pointer_cmp_lt_dep()}, and
\co{pointer_cmp_le_dep()},
so that as long as the first argument to a relational operator is of type
\co{class depending_ptr<T>},
pointers may be safely compared without risk of breaking dependency
chains.\footnote{
	That said, in the prototype implementation,
	these are not intrinsics, but rather separately compiled functions.
	In the absence of link-time optimizations, separate compilation
	preserves dependency chains in most implementations.}
In addition, the operators that cannot be safely applied to
dependency-bearing pointers have been deleted.\footnote{
	The number of pointer-tagging algorithms should motivate
	allowing bitwise operations on dependency-bearing
	pointers, but this should be handled separately.}

\begin{figure}[tbp]
{ \scriptsize
\begin{verbatim}
 1 void thread0(void)
 2 {
 3   struct rcutest *p;
 4
 5   p = (struct rcutest *)malloc(sizeof(*p));
 6   assert(p);
 7   p->a = 42;
 8   assert(p->a != 43);
 9   rcu_store_release(&gp, p);
10 }
11
12 void thread1(void)
13 {
14   depending_ptr<struct rcutest> p;
15
16   p = rcu_consume(&gp);
17   if (p)
18     p->a = 43;
19 }
\end{verbatim}
}
\caption{Template: Simple Case}
\label{fig:Template: Simple Case}
\end{figure}

\begin{figure}[tbp]
{ \scriptsize
\begin{verbatim}
 1 void thread0(void)
 2 {
 3   struct rcutest *p;
 4
 5   p = (struct rcutest *)malloc(sizeof(*p));
 6   assert(p);
 7   p->a = 42;
 8   rcu_store_release(&gp, p);
 9 }
10
11 void
12 thread1_help(depending_ptr<struct rcutest> q)
13 {
14   if (q)
15     assert(q->a == 42);
16 }
17
18 void thread1(void)
19 {
20   depending_ptr<struct rcutest> p;
21
22   p = rcu_consume(&gp);
23   thread1_help(p);
24 }
\end{verbatim}
}
\caption{Template: In via Function Parameter}
\label{fig:Template: In via Function Parameter}
\end{figure}

\begin{figure}[tbp]
{ \scriptsize
\begin{verbatim}
 1 void thread0(void)
 2 {
 3   struct rcutest *p;
 4
 5   p = (struct rcutest *)malloc(sizeof(*p));
 6   assert(p);
 7   p->a = 42;
 8   rcu_store_release(&gp, p);
 9 }
10
11 depending_ptr<struct rcutest> thread1_help(void)
12 {
13   return rcu_consume(&gp);
14 }
15
16 void thread1(void)
17 {
18   depending_ptr<struct rcutest> p;
19
20   p = thread1_help();
21   if (p)
22     assert(p->a == 42);
23 }
\end{verbatim}
}
\caption{Template: Out via Function Return}
\label{fig:Template: Out via Function Return}
\end{figure}

\begin{figure}[tbp]
{ \scriptsize
\begin{verbatim}
 1 void thread0(void)
 2 {
 3   struct rcutest *p;
 4
 5   p = (struct rcutest *)malloc(sizeof(*p));
 6   assert(p);
 7   p->a = 42;
 8   rcu_store_release(&gp, p);
 9
10   p = (struct rcutest *)malloc(sizeof(*p));
11   assert(p);
12   p->a = 43;
13   rcu_store_release(&gsgp, p);
14 }
15
16 depending_ptr<struct rcutest> 
17 thread1_help(depending_ptr<struct rcutest> p)
18 {
19   if (p)
20     assert(p->a == 42);
21   return rcu_consume(&gsgp);
22 }
23
24 void thread1(void)
25 {
26   depending_ptr<struct rcutest> p;
27
28   p = rcu_consume(&gp);
29   p = thread1_help(p);
30   if (p)
31     assert(p->a == 43);
32 }
\end{verbatim}
}
\caption{Template: In and Out, But Different Chains}
\label{fig:Template: In and Out, But Different Chains}
\end{figure}

\begin{figure}[tbp]
{ \scriptsize
\begin{verbatim}
 1 void thread0(void)
 2 {
 3   struct rcutest *p;
 4
 5   p = (struct rcutest *)malloc(sizeof(*p));
 6   assert(p);
 7   p->a = 42;
 8   rcu_store_release(&gp, p);
 9 }
10
11 void
12 thread1_help1(depending_ptr<struct rcutest> q)
13 {
14   if (q)
15     assert(q->a == 42);
16 }
17
18 void
19 thread1_help2(depending_ptr<struct rcutest> q)
20 {
21   if (q)
22     assert(q->a != 43);
23 }
24
25 void thread1(void)
26 {
27   depending_ptr<struct rcutest> p;
28
29   p = rcu_consume(&gp);
30   thread1_help1(p);
31   thread1_help2(p);
32 }
\end{verbatim}
}
\caption{Template: Chain Fanning Out}
\label{fig:Template: Chain Fanning Out}
\end{figure}

\begin{figure}[tbp]
{ \scriptsize
\begin{verbatim}
 1 void thread0(void)
 2 {
 3   struct rcutest *p;
 4   struct rcutest1 *p1;
 5
 6   p = (struct rcutest *)malloc(sizeof(*p));
 7   assert(p);
 8   p->a = 42;
 9   rcu_store_release(&gp, p);
10
11   p1 = (struct rcutest *)malloc(sizeof(*p1));
12   assert(p1);
13   p1->a = 41;
14   p1->rt.a = 42;
15   rcu_store_release(&g1p, p1);
16 }
17
18 void
19 thread1_help(depending_ptr<struct rcutest> q)
20 {
21   if (q)
22     assert(q->a == 42);
23 }
24
25 void thread1(void)
26 {
27   depending_ptr<struct rcutest> p;
28
29   p = rcu_consume(&gp);
30   thread1_help(p);
31 }
32
33 void thread2(void)
34 {
35   depending_ptr<struct rcutest1> p1;
36
37   p1 = rcu_consume(&g1p);
38   thread1_help(&p1->rt);
39 }
\end{verbatim}
}
\caption{Template: Chain Fanning In}
\label{fig:Template: Chain Fanning In}
\end{figure}

\begin{figure}[tbp]
{ \scriptsize
\begin{verbatim}
 1 void thread0(void)
 2 {
 3   struct rcutest *p;
 4   struct rcutest1 *p1;
 5
 6   p = (struct rcutest *)malloc(sizeof(*p));
 7   assert(p);
 8   p->a = 42;
 9   p->b = 43;
10   rcu_store_release(&gp, p);
11
12   p1 = (struct rcutest *)malloc(sizeof(*p1));
13   assert(p1);
14   p1->a = 41;
15   p1->rt.a = 42;
16   p1->rt.b = 43;
17   rcu_store_release(&g1p, p1);
18 }
19
20 void
21 thread1a_help(depending_ptr<struct rcutest> q)
22 {
23   assert(q->a == 42);
24 }
25
26 void
27 thread1b_help(depending_ptr<struct rcutest> q)
28 {
29   assert(q->b == 43);
30 }
31
32 void
33 thread1_help(depending_ptr<struct rcutest> q)
34 {
35   if (q) {
36     thread1a_help(q);
37     thread1b_help(q);
38   }
39 }
40
41 void thread1(void)
42 {
43   depending_ptr<struct rcutest> p;
44
45   p = rcu_consume(&gp);
46   thread1_help(p);
47 }
48
49 void thread2(void)
50 {
51   depending_ptr<struct rcutest1> p1;
52
53   p1 = rcu_consume(&g1p);
54   thread1_help(&p1->rt);
55 }
\end{verbatim}
}
\caption{Template: Chain Fanning In and Out}
\label{fig:Template: Chain Fanning In and Out}
\end{figure}

\begin{figure}[tbp]
{ \scriptsize
\begin{verbatim}
 1 void thread0(void)
 2 {
 3   struct rcutest *p;
 4   struct rcutest1 *p1;
 5
 6   p = (struct rcutest *)malloc(sizeof(*p));
 7   assert(p);
 8   p->a = 42;
 9   p->b = 43;
10   rcu_store_release(&gp, p);
11
12   p1 = (struct rcutest *)malloc(sizeof(*p1));
13   assert(p1);
14   p1->a = 41;
15   p1->rt.a = 42;
16   p1->rt.b = 43;
17   rcu_store_release(&g1p, p1);
18 }
19
20 #ifdef FOO
21 void
22 thread1a_help(depending_ptr<struct rcutest> q)
23 {
24   assert(q->a == 42);
25 }
26 #endif
27
28 void
29 thread1b_help(depending_ptr<struct rcutest> q)
30 {
31   assert(q->b == 43);
32 }
33
34 void
35 thread1_help(depending_ptr<struct rcutest> q)
36 {
37   if (q) {
38 #ifdef FOO
39     thread1a_help(q);
40 #endif
41     thread1b_help(q);
42   }
43 }
44
45 void thread1(void)
46 {
47   depending_ptr<struct rcutest> p;
48
49   p = rcu_consume(&gp);
50   thread1_help(p);
51 }
52
53 void thread2(void)
54 {
55   depending_ptr<struct rcutest1> p1;
56
57   p1 = rcu_consume(&g1p);
58   thread1_help(&p1->rt);
59 }
\end{verbatim}
}
\caption{Template: Conditional Compilation of Chain Endpoints}
\label{fig:Template: Conditional Compilation of Chain Endpoints}
\end{figure}

\begin{figure}[tbp]
{ \scriptsize
\begin{verbatim}
 1 void thread0(void)
 2 {
 3   struct rcutest *p;
 4
 5   p = (struct rcutest *)malloc(sizeof(*p));
 6   assert(p);
 7   p->a = 42;
 8   assert(p->a != 43);
 9   rcu_store_release(&gp, p);
10 }
11
12 void thread1(void)
13 {
14   depending_ptr<struct rcutest> p;
15
16   p = rcu_consume(&gp);
17   if (p) {
18     assert(p->a >= 42);
19     spin_lock(&p->lock);
20     p = std::kill_dependency(p);
21     p->a++;
22     spin_unlock(&p->lock);
23   }
24 }
\end{verbatim}
}
\caption{Template: Handoff to Locking}
\label{fig:Template: Handoff to Locking}
\end{figure}

Figures~\ref{fig:Template: Simple Case}--\ref{fig:Template: Handoff to Locking}
show how templates can be applied to each of the examples
introduced in Section~\ref{sec:Representative Use Cases}.
As with the object-modifier approach in
Section~\ref{sec:Object Modifier},
these changes are straightforward markings of local variables, function
parameters, and return-value types.

Full source code for a prototype implementation
(and for this paper) may be downloaded from
{\small \co{https://github.com/paulmckrcu/2016markconsume.git}}.
