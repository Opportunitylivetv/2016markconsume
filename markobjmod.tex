This approach uses a keyword that does not participate in type checking,
for example, a \co{_Carries_dependency} keyword.
This might be treated in a manner similar to a storage class.
It need not necessarily interact with the type system.

\begin{figure}[tbp]
{ \scriptsize
\begin{verbatim}
 1 void thread0(void)
 2 {
 3   struct rcutest *p;
 4
 5   p = (struct rcutest *)malloc(sizeof(*p));
 6   assert(p);
 7   p->a = 42;
 8   assert(p->a != 43);
 9   rcu_assign_pointer(gp, p);
10 }
11
12 void thread1(void)
13 {
14   struct rcutest _Carries_dependency *p;
15
16   p = rcu_dereference(gp);
17   if (p)
18     p->a = 43;
19 }
\end{verbatim}
}
\caption{Object Modifier: Simple Case}
\label{fig:Object Modifier: Simple Case}
\end{figure}

\begin{figure}[tbp]
{ \scriptsize
\begin{verbatim}
 1 void thread0(void)
 2 {
 3   struct rcutest *p;
 4
 5   p = (struct rcutest *)malloc(sizeof(*p));
 6   assert(p);
 7   p->a = 42;
 8   rcu_assign_pointer(gp, p);
 9 }
10
11 void
12 thread1_help(struct rcutest _Carries_dependency *q)
13 {
14   if (q)
15     assert(q->a == 42);
16 }
17
18 void thread1(void)
19 {
20   struct rcutest _Carries_dependency *p;
21
22   p = rcu_dereference(gp);
23   thread1_help(p);
24 }
\end{verbatim}
}
\caption{Object Modifier: In via Function Parameter}
\label{fig:Object Modifier: In via Function Parameter}
\end{figure}

\begin{figure}[tbp]
{ \scriptsize
\begin{verbatim}
 1 void thread0(void)
 2 {
 3   struct rcutest *p;
 4
 5   p = (struct rcutest *)malloc(sizeof(*p));
 6   assert(p);
 7   p->a = 42;
 8   rcu_assign_pointer(gp, p);
 9 }
10
11 struct rcutest _Carries_dependency *thread1_help(void)
12 {
13   return rcu_dereference(gp);
14 }
15
16 void thread1(void)
17 {
18   struct rcutest _Carries_dependency *p;
19
20   p = thread1_help();
21   if (p)
22     assert(p->a == 42);
23 }
\end{verbatim}
}
\caption{Object Modifier: Out via Function Return}
\label{fig:Object Modifier: Out via Function Return}
\end{figure}

\begin{figure}[tbp]
{ \scriptsize
\begin{verbatim}
 1 void thread0(void)
 2 {
 3   struct rcutest *p;
 4
 5   p = (struct rcutest *)malloc(sizeof(*p));
 6   assert(p);
 7   p->a = 42;
 8   rcu_assign_pointer(gp, p);
 9
10   p = (struct rcutest *)malloc(sizeof(*p));
11   assert(p);
12   p->a = 43;
13   rcu_assign_pointer(gsgp, p);
14 }
15
16 struct rcutest _Carries_dependency *
17 thread1_help(struct rcutest _Carries_dependency *p)
18 {
19   if (p)
20     assert(p->a == 42);
21   return rcu_dereference(gsgp);
22 }
23
24 void thread1(void)
25 {
26   struct rcutest _Carries_dependency *p;
27
28   p = rcu_dereference(gp);
29   p = thread1_help(p);
30   if (p)
31     assert(p->a == 43);
32 }
\end{verbatim}
}
\caption{Object Modifier: In and Out, But Different Chains}
\label{fig:Object Modifier: In and Out, But Different Chains}
\end{figure}

\begin{figure}[tbp]
{ \scriptsize
\begin{verbatim}
 1 void thread0(void)
 2 {
 3   struct rcutest *p;
 4
 5   p = (struct rcutest *)malloc(sizeof(*p));
 6   assert(p);
 7   p->a = 42;
 8   rcu_assign_pointer(gp, p);
 9 }
10
11 void
12 thread1_help1(struct rcutest _Carries_dependency *q)
13 {
14   if (q)
15     assert(q->a == 42);
16 }
17
18 void
19 thread1_help2(struct rcutest _Carries_dependency *q)
20 {
21   if (q)
22     assert(q->a != 43);
23 }
24
25 void thread1(void)
26 {
27   struct rcutest _Carries_dependency *p;
28
29   p = rcu_dereference(gp);
30   thread1_help1(p);
31   thread1_help2(p);
32 }
\end{verbatim}
}
\caption{Object Modifier: Chain Fanning Out}
\label{fig:Object Modifier: Chain Fanning Out}
\end{figure}

\begin{figure}[tbp]
{ \scriptsize
\begin{verbatim}
 1 void thread0(void)
 2 {
 3   struct rcutest *p;
 4   struct rcutest1 *p1;
 5
 6   p = (struct rcutest *)malloc(sizeof(*p));
 7   assert(p);
 8   p->a = 42;
 9   rcu_assign_pointer(gp, p);
10
11   p1 = (struct rcutest *)malloc(sizeof(*p1));
12   assert(p1);
13   p1->a = 41;
14   p1->rt.a = 42;
15   rcu_assign_pointer(g1p, p1);
16 }
17
18 void
19 thread1_help(struct rcutest _Carries_dependency *q)
20 {
21   if (q)
22     assert(q->a == 42);
23 }
24
25 void thread1(void)
26 {
27   struct rcutest _Carries_dependency *p;
28
29   p = rcu_dereference(gp);
30   thread1_help(p);
31 }
32
33 void thread2(void)
34 {
35   struct rcutest1 _Carries_dependency *p1;
36
37   p1 = rcu_dereference(g1p);
38   thread1_help(&p1->rt);
39 }
\end{verbatim}
}
\caption{Object Modifier: Chain Fanning In}
\label{fig:Object Modifier: Chain Fanning In}
\end{figure}

\begin{figure}[tbp]
{ \scriptsize
\begin{verbatim}
 1 void thread0(void)
 2 {
 3   struct rcutest *p;
 4   struct rcutest1 *p1;
 5
 6   p = (struct rcutest *)malloc(sizeof(*p));
 7   assert(p);
 8   p->a = 42;
 9   p->b = 43;
10   rcu_assign_pointer(gp, p);
11
12   p1 = (struct rcutest *)malloc(sizeof(*p1));
13   assert(p1);
14   p1->a = 41;
15   p1->rt.a = 42;
16   p1->rt.b = 43;
17   rcu_assign_pointer(g1p, p1);
18 }
19
20 void
21 thread1a_help(struct rcutest _Carries_dependency *q)
22 {
23   assert(q->a == 42);
24 }
25
26 void
27 thread1b_help(struct rcutest _Carries_dependency *q)
28 {
29   assert(q->b == 43);
30 }
31
32 void
33 thread1_help(struct rcutest _Carries_dependency *q)
34 {
35   if (q) {
36     thread1a_help(q);
37     thread1b_help(q);
38   }
39 }
40
41 void thread1(void)
42 {
43   struct rcutest _Carries_dependency *p;
44
45   p = rcu_dereference(gp);
46   thread1_help(p);
47 }
48
49 void thread2(void)
50 {
51   struct rcutest1 _Carries_dependency *p1;
52
53   p1 = rcu_dereference(g1p);
54   thread1_help(&p1->rt);
55 }
\end{verbatim}
}
\caption{Object Modifier: Chain Fanning In and Out}
\label{fig:Object Modifier: Chain Fanning In and Out}
\end{figure}

\begin{figure}[tbp]
{ \scriptsize
\begin{verbatim}
 1 void thread0(void)
 2 {
 3   struct rcutest *p;
 4   struct rcutest1 *p1;
 5
 6   p = (struct rcutest *)malloc(sizeof(*p));
 7   assert(p);
 8   p->a = 42;
 9   p->b = 43;
10   rcu_assign_pointer(gp, p);
11
12   p1 = (struct rcutest *)malloc(sizeof(*p1));
13   assert(p1);
14   p1->a = 41;
15   p1->rt.a = 42;
16   p1->rt.b = 43;
17   rcu_assign_pointer(g1p, p1);
18 }
19
20 #ifdef FOO
21 void
22 thread1a_help(struct rcutest _Carries_dependency *q)
23 {
24   assert(q->a == 42);
25 }
26 #endif
27
28 void
29 thread1b_help(struct rcutest _Carries_dependency *q)
30 {
31   assert(q->b == 43);
32 }
33
34 void
35 thread1_help(struct rcutest _Carries_dependency *q)
36 {
37   if (q) {
38 #ifdef FOO
39     thread1a_help(q);
40 #endif
41     thread1b_help(q);
42   }
43 }
44
45 void thread1(void)
46 {
47   struct rcutest _Carries_dependency *p;
48
49   p = rcu_dereference(gp);
50   thread1_help(p);
51 }
52
53 void thread2(void)
54 {
55   struct rcutest1 _Carries_dependency *p1;
56
57   p1 = rcu_dereference(g1p);
58   thread1_help(&p1->rt);
59 }
\end{verbatim}
}
\caption{Object Modifier: Conditional Compilation of Chain Endpoints}
\label{fig:Object Modifier: Conditional Compilation of Chain Endpoints}
\end{figure}

\begin{figure}[tbp]
{ \scriptsize
\begin{verbatim}
 1 void thread0(void)
 2 {
 3   struct rcutest *p;
 4
 5   p = (struct rcutest *)malloc(sizeof(*p));
 6   assert(p);
 7   p->a = 42;
 8   assert(p->a != 43);
 9   rcu_assign_pointer(gp, p);
10 }
11
12 void thread1(void)
13 {
14   struct rcutest _Carries_dependency *p;
15
16   p = rcu_dereference(gp);
17   if (p) {
18     assert(p->a >= 42);
19     spin_lock(&p->lock);
20     p = std::kill_dependency(p);
21     p->a++;
22     spin_unlock(&p->lock);
23   }
24 }
\end{verbatim}
}
\caption{Object Modifier: Handoff to Locking}
\label{fig:Object Modifier: Handoff to Locking}
\end{figure}

Figures~\ref{fig:Object Modifier: Simple Case}--\ref{fig:Object Modifier: Handoff to Locking}
show how object modifiers can be applied to each of the examples
introduced in Section~\ref{sec:Representative Use Cases}.
These changes are straightforward markings of local variables, function
parameters, and return-value types.
Object modifiers therefore easily support the use cases in the Linux
kernel.\footnote{
	Give or take a strong distaste for any sort of marking scheme
	on the part of numerous Linux-kernel community members.}
