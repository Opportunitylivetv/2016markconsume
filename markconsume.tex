\documentclass[letterpaper,twocolumn,10pt]{article}
\usepackage{epsfig,endnotes}
%\usepackage{usenix,epsfig}

%%%%%%%%%%%%%%%%%%%%%%%%%%%%%%%%%%%%%%%%%%%%%%%%%%%%%%%%%%%%%%%%%%%%%%
\usepackage{subfig}
\usepackage{fixltx2e}
\usepackage{url}        % \url{} command with good linebreaks
\usepackage{amssymb,amsmath}
\usepackage{graphicx}
\usepackage{enumerate}
\usepackage{listings}
\usepackage{xspace}
\usepackage{xcolor}
\lstset{basicstyle=\ttfamily}
\usepackage[bookmarks=true,bookmarksnumbered=true,pdfborder={0 0 0}]{hyperref}

% Avoid widows and orphans
\widowpenalty=500
\clubpenalty=500

%%%%%%%%%%%%%%%%%%%%%%%%%%%%%%%%%%%%%%%%%%%%%%%%%%%%%%%%%%%%%%%%%%%%%%

\begin{document}

\lstset{
	literate={\_}{}{0\discretionary{\_}{}{\_}}%
}
\newcommand{\co}[1]{\lstinline[breaklines=yes,breakatwhitespace=yes]{#1}}

\title{DNNNNR0: Marking {\tt memory\_order\_consume} Dependency Chains}

\author{
{\bf Doc. No.: } WG21/DNNNNR0 \\
{\bf Date: } 2016-03-08 \\
{\bf Reply to: } Paul E. McKenney, Torvald Riegel, Jeff Preshing, \\
	Hans Boehm, Clark Nelson, Olivier Giroux, Lawrence Crowl, \\
	and JF Bastien \\
{\bf Email: } paulmck@linux.vnet.ibm.com, triegel@redhat.com,
jeff@preshing.com \\
boehm@acm.org, clark.nelson@intel.com, OGiroux@nvidia.com, \\
Lawrence@Crowl.org, and cxx@jfbastien.com \\ ~ \\
{\bf Other contributors: }
	Alec Teal,
	David Howells,
	David Lang,
	George Spelvin, \\
	Jeff Law,
	Joseph S. Myers,
	Linus Torvalds,
	Mark Batty,
	Michael Matz,
	Peter Sewell, \\
	Peter Zijlstra,
	Ramana Radhakrishnan,
	Richard Biener,
	Will Deacon,
	Faisal Vali, \\
	Behan Webster,
	Tony Tye,
	Thomas Koeppe,
	Jens Maurer,
	... \\
} % end author

% Use the following at camera-ready time to suppress page numbers.
% Comment it out when you first submit the paper for review.
%\thispagestyle{empty}

\pagestyle{myheadings}
\markright{WG21/D0098R1}

\maketitle

%%%%%%%%%%%%%%%%%%%%%%%%%%%%%%%%%%%%%%%%%%%%%%%%%%%%%%%%%%%%%%%%%%%%%%

This document is based in part on WG21/D0098R1, extracting the
alternatives for marking dependency chains (each headed by a
\co{memory_order_consume}~\cite{RichardSmith2015N4527} load).
It also adds a few additional alternatives based on discussions
at the March 2016 meeting in Jacksonville, Florida,

% A detailed change log appears starting on
% page~\pageref{sec:Change Log}.

\section{Introduction}
\label{sec:Introduction}

Spirited discussions of \co{memory_order_consume}
at the Jacksonville meeting resulted in a few of items of agreement:

\begin{enumerate}
\item	Dependency chains should be restricted to pointers.
	Please note that this excludes not only the troublesome objects
	of integral type, but also accesses to static members of classes.
\item	Unmarked code can be handled by having the implementation
	behave as if markings had been supplied in all locations that
	could reasonably be marked.
	This allows natural handling of dependencies in unmarked code.
	This behavior should be controlled by a compiler flag.
	Such a flag is of course outside of the standard.
\item	Software artifacts that are built standalone (such as the Linux
	kernel and numerous embedded projects) can reasonably use
	unmarked dependency chains.
	In contrast, software artifacts that are expected to dynamically link
	against standard libraries seem likely to need to mark their
	dependency chains.
\item	Discussions involving marking of library APIs have been
	set aside for the moment, and so this document does not address
	this point.
\end{enumerate}

These points result in three known valid ways of handling
\co{memory_order_consume}:

\begin{enumerate}
\item	Ignore the markings and promote \co{memory_order_consume}
	to \co{memory_order_acquire}, as is current practice.
\item	Ignore the markings, demote \co{memory_order_consume} to
	\co{memory_order_relaxed}, and suppress troublesome
	optimizations of pointers.
	However, there was some difficulty in arriving at a precise
	definition of ``troublesome''.
\item	Demote \co{memory_order_consume} to \co{memory_order_relaxed}
	and suppress troublesome optimizations of marked pointers.
	The fact that such optimizations need not be suppressed
	for unmarked pointers means that a much more conservative
	definition of ``troublesome'' is feasible, thus reducing
	the need for precision.
	Note that pointer comparisons will still break dependency chains
	in some cases, unless the comparisons were carried out using
	proposed dependency-preserving pointer-comparison intrinsics.
	Note further that the template-based method described in
	Section~\ref{sec:Template}
	uses operator overloading so that the usual relational
	operators invoke these intrinsics.
\end{enumerate}

However, a number of ways of marking dependency chains have been
proposed and there was nothing resembling any sort of agreement
on which should be used.
This paper therefore catalogs approaches to marking dependency chains,
and evaluates each against a set of representative use cases.

\section{Representative Use Cases}
\label{sec:Representative Use Cases}

\begin{figure}[tbp]
{ \scriptsize
\begin{verbatim}
 1 struct rcutest {
 2   int a;
 3   int b;
 4   int c;
 5   spinlock_t lock;
 6 };
 7
 8 struct rcutest1 {
 9   int a;
10   struct rcutest rt;
11 };
12
13 std::atomic<rcutest *> gp;
14 std::atomic<rcutest1 *> g1p;
15 std::atomic<int *> gip;
16 struct rcutest *gslp; /* Global scope, local usage. */
17 std::atomic<rcutest *> gsgp;
18
19 #define rcu_assign_pointer(p, v) \
20   atomic_store_explicit(&(p), (v),
21                         std::memory_order_release);
22 #define rcu_dereference(p) \
23   atomic_load_explicit(&(p), std::memory_order_consume);
\end{verbatim}
}
\caption{Common Definitions}
\label{fig:Common Definitions}
\end{figure}

This section uses the common definitions shown in
Figure~\ref{fig:Common Definitions}
to discuss the use cases in the following list:

\begin{enumerate}
\item	Simple case.
\item	Function in via parameter.
\item	Function out via return value.
\item	Function both in and out, but different chains.
\item	Dependency chain fanning out.
\item	Dependency chain fanning in.
\item	Dependency chain fanning both in and out.
\item	Conditional compilation of endpoint accesses.
\item	Examples involving handoff to locking.
\end{enumerate}

Each of the above use cases is covered in one of the following sections,
followed by a discussion of evaluation criteria.

\subsection{Simple Case}
\label{sec:Simple Case}

\begin{figure}[tbp]
{ \scriptsize
\begin{verbatim}
 1 void thread0(void)
 2 {
 3   struct rcutest *p;
 4
 5   p = (struct rcutest *)malloc(sizeof(*p));
 6   assert(p);
 7   p->a = 42;
 8   assert(p->a != 43);
 9   rcu_assign_pointer(gp, p);
10 }
11
12 void thread1(void)
13 {
14   struct rcutest *p;
15
16   p = rcu_dereference(gp);
17   if (p)
18     p->a = 43;
19 }
\end{verbatim}
}
\caption{Simple Case}
\label{fig:Simple Case}
\end{figure}

The simple case is shown in
Figure~\ref{fig:Simple Case}.
Here, the dependency chain extends from line~16 through line~18,
where it terminates.
Given the simplicity and compactness of this example, any reasonable
proposal should handle this example simply and naturally.

\subsection{In via Function Parameter}
\label{sec:In via Function Parameter}

\begin{figure}[tbp]
{ \scriptsize
\begin{verbatim}
 1 void thread0(void)
 2 {
 3   struct rcutest *p;
 4
 5   p = (struct rcutest *)malloc(sizeof(*p));
 6   assert(p);
 7   p->a = 42;
 8   rcu_assign_pointer(gp, p);
 9 }
10
11 void
12 thread1_help(struct rcutest *q)
13 {
14   if (q)
15     assert(q->a == 42);
16 }
17
18 void thread1(void)
19 {
20   struct rcutest *p;
21
22   p = rcu_dereference(gp);
23   thread1_help(p);
24 }
\end{verbatim}
}
\caption{In via Function Parameter}
\label{fig:In via Function Parameter}
\end{figure}

Figure~\ref{fig:In via Function Parameter}
shows an example dependency chain that begins at line~22,
enters function \co{thread1_help()} at line~23,
and then extending from line~12 to line~15 in the called function.
This is a common encapsulation technique.

\subsection{Out via Function Return}
\label{sec:Out via Function Return}

\begin{figure}[tbp]
{ \scriptsize
\begin{verbatim}
 1 void thread0(void)
 2 {
 3   struct rcutest *p;
 4
 5   p = (struct rcutest *)malloc(sizeof(*p));
 6   assert(p);
 7   p->a = 42;
 8   rcu_assign_pointer(gp, p);
 9 }
10
11 struct rcutest *thread1_help(void)
12 {
13   return rcu_dereference(gp);
14 }
15
16 void thread1(void)
17 {
18   struct rcutest *p;
19
20   p = thread1_help();
21   if (p)
22     assert(p->a == 42);
23 }
\end{verbatim}
}
\caption{Out via Function Return}
\label{fig:Out via Function Return}
\end{figure}

Figure~\ref{fig:Out via Function Return}
shows a dependency chain exiting a function.
It starts at line~21, is returned to line~20, and
terminates on line~22.
This is also a common encapsulation technique.

\subsection{In and Out, But Different Chains}
\label{sec:In and Out, But Different Chains}

\begin{figure}[tbp]
{ \scriptsize
\begin{verbatim}
 1 void thread0(void)
 2 {
 3   struct rcutest *p;
 4
 5   p = (struct rcutest *)malloc(sizeof(*p));
 6   assert(p);
 7   p->a = 42;
 8   rcu_assign_pointer(gp, p);
 9
10   p = (struct rcutest *)malloc(sizeof(*p));
11   assert(p);
12   p->a = 43;
13   rcu_assign_pointer(gsgp, p);
14 }
15
16 struct rcutest *thread1_help(struct rcutest *p)
17 {
18   if (p)
19     assert(p->a == 42);
20   return rcu_dereference(gsgp);
21 }
22
23 void thread1(void)
24 {
25   struct rcutest *p;
26
27   p = rcu_dereference(gp);
28   p = thread1_help(p);
29   if (p)
30     assert(p->a == 43);
31 }
\end{verbatim}
}
\caption{In and Out, But Different Chains}
\label{fig:In and Out, But Different Chains}
\end{figure}

Figure~\ref{fig:In and Out, But Different Chains}
shows an example where a dependency chain enters a function
(\co{thread1_help()} on lines~16-21)
and a dependency chain leaves that same function,
but where they are different chains.

\subsection{Chain Fanning Out}
\label{sec:Chain Fanning Out}

\begin{figure}[tbp]
{ \scriptsize
\begin{verbatim}
 1 void thread0(void)
 2 {
 3   struct rcutest *p;
 4
 5   p = (struct rcutest *)malloc(sizeof(*p));
 6   assert(p);
 7   p->a = 42;
 8   rcu_assign_pointer(gp, p);
 9 }
10
11 void
12 thread1_help1(struct rcutest *q)
13 {
14   if (q)
15     assert(q->a == 42);
16 }
17
18 void
19 thread1_help2(struct rcutest *q)
20 {
21   if (q)
22     assert(q->a != 43);
23 }
24
25 void thread1(void)
26 {
27   struct rcutest *p;
28
29   p = rcu_dereference(gp);
30   thread1_help1(p);
31   thread1_help2(p);
32 }
\end{verbatim}
}
\caption{Chain Fanning Out}
\label{fig:Chain Fanning Out}
\end{figure}

Figure~\ref{fig:Chain Fanning Out}
shows a dependency chain fanning out, courtesy of the \co{thread1()}
function's calls to \co{thread1_help1()} and \co{thread1_help2()}
on lines~30 and~31.
This is a common pattern in the Linux kernel, as it supports
abstraction of data structures, for example, allowing common RCU-protected
data structures to be aggregated into a larger RCU-protected data
structure.
In this scenario, \co{thread1_help1()} might implement one type of
RCU-protected structure and \co{thread1_help2()} might implement
another.

\subsection{Chain Fanning In}
\label{sec:Chain Fanning In}

\begin{figure}[tbp]
{ \scriptsize
\begin{verbatim}
 1 void thread0(void)
 2 {
 3   struct rcutest *p;
 4   struct rcutest1 *p1;
 5
 6   p = (struct rcutest *)malloc(sizeof(*p));
 7   assert(p);
 8   p->a = 42;
 9   rcu_assign_pointer(gp, p);
10
11   p1 = (struct rcutest *)malloc(sizeof(*p1));
12   assert(p1);
13   p1->a = 41;
14   p1->rt.a = 42;
15   rcu_assign_pointer(g1p, p1);
16 }
17
18 void
19 thread1_help(struct rcutest *q)
20 {
21   if (q)
22     assert(q->a == 42);
23 }
24
25 void thread1(void)
26 {
27   struct rcutest *p;
28
29   p = rcu_dereference(gp);
30   thread1_help(p);
31 }
32
33 void thread2(void)
34 {
35   struct rcutest1 *p1;
36
37   p1 = rcu_dereference(g1p);
38   thread1_help(&p1->rt);
39 }
\end{verbatim}
}
\caption{Chain Fanning In}
\label{fig:Chain Fanning In}
\end{figure}

Figure~\ref{fig:Chain Fanning In}
demonstrates different dependency chains fanning into the same function,
in this case \co{thread1_help()}, from lines~29 and~37.
This fanning-in is also used to support abstraction, for example,
allowing a given implementation of an RCU-protected data structure
to be aggregated into several different data structures.

\subsection{Chain Fanning In and Out}
\label{sec:Chain Fanning In and Out}

\begin{figure}[tbp]
{ \scriptsize
\begin{verbatim}
 1 void thread0(void)
 2 {
 3   struct rcutest *p;
 4   struct rcutest1 *p1;
 5
 6   p = (struct rcutest *)malloc(sizeof(*p));
 7   assert(p);
 8   p->a = 42;
 9   p->b = 43;
10   rcu_assign_pointer(gp, p);
11
12   p1 = (struct rcutest *)malloc(sizeof(*p1));
13   assert(p1);
14   p1->a = 41;
15   p1->rt.a = 42;
16   p1->rt.b = 43;
17   rcu_assign_pointer(g1p, p1);
18 }
19
20 void
21 thread1a_help(struct rcutest *q)
22 {
23   assert(q->a == 42);
24 }
25
26 void
27 thread1b_help(struct rcutest *q)
28 {
29   assert(q->b == 43);
30 }
31
32 void
33 thread1_help(struct rcutest *q)
34 {
35   if (q) {
36     thread1a_help(q);
37     thread1b_help(q);
38   }
39 }
40
41 void thread1(void)
42 {
43   struct rcutest *p;
44
45   p = rcu_dereference(gp);
46   thread1_help(p);
47 }
48
49 void thread2(void)
50 {
51   struct rcutest1 *p1;
52
53   p1 = rcu_dereference(g1p);
54   thread1_help(&p1->rt);
55 }
\end{verbatim}
}
\caption{Chain Fanning In and Out}
\label{fig:Chain Fanning In and Out}
\end{figure}

Figure~\ref{fig:Chain Fanning In and Out}
shows dependency chains fanning both in and out, starting
at lines~45 and~53, fanning into \co{thread1_help()}, and
fanning out again at the call to \co{thread1a_help()} on
line~36 and to \co{thread1b_help()} on line~37.
This combination permits composition of the types of abstraction
described in
Sections~\ref{sec:Chain Fanning Out}
and~\ref{sec:Chain Fanning In}.

\subsection{Conditional Compilation of Chain Endpoints}
\label{sec:Conditional Compilation of Chain Endpoints}

\begin{figure}[tbp]
{ \scriptsize
\begin{verbatim}
 1 void thread0(void)
 2 {
 3   struct rcutest *p;
 4   struct rcutest1 *p1;
 5
 6   p = (struct rcutest *)malloc(sizeof(*p));
 7   assert(p);
 8   p->a = 42;
 9   p->b = 43;
10   rcu_assign_pointer(gp, p);
11
12   p1 = (struct rcutest *)malloc(sizeof(*p1));
13   assert(p1);
14   p1->a = 41;
15   p1->rt.a = 42;
16   p1->rt.b = 43;
17   rcu_assign_pointer(g1p, p1);
18 }
19
20 #ifdef FOO
21 void
22 thread1a_help(struct rcutest *q)
23 {
24   assert(q->a == 42);
25 }
26 #endif
27
28 void
29 thread1b_help(struct rcutest *q)
30 {
31   assert(q->b == 43);
32 }
33
34 void
35 thread1_help(struct rcutest *q)
36 {
37   if (q) {
38 #ifdef FOO
39     thread1a_help(q);
40 #endif
41     thread1b_help(q);
42   }
43 }
44
45 void thread1(void)
46 {
47   struct rcutest *p;
48
49   p = rcu_dereference(gp);
50   thread1_help(p);
51 }
52
53 void thread2(void)
54 {
55   struct rcutest1 *p1;
56
57   p1 = rcu_dereference(g1p);
58   thread1_help(&p1->rt);
59 }
\end{verbatim}
}
\caption{Conditional Compilation of Chain Endpoints}
\label{fig:Conditional Compilation of Chain Endpoints}
\end{figure}

Although the C preprocessor does not necessarily have the best reputation
among the various aspects of either C or C++, it is true that it is
always there when you need it.
Figure~\ref{fig:Conditional Compilation of Chain Endpoints}
applies conditional compilation to
Figure~\ref{fig:Chain Fanning In and Out},
so that portions of the dependency chain can come and go, depending
on the value of the C-preprocessor macro \co{FOO}.

\subsection{Handoff to Locking}
\label{sec:Handoff to Locking}

\begin{figure}[tbp]
{ \scriptsize
\begin{verbatim}
 1 void thread0(void)
 2 {
 3   struct rcutest *p;
 4
 5   p = (struct rcutest *)malloc(sizeof(*p));
 6   assert(p);
 7   p->a = 42;
 8   assert(p->a != 43);
 9   rcu_assign_pointer(gp, p);
10 }
11
12 void thread1(void)
13 {
14   struct rcutest *p;
15
16   p = rcu_dereference(gp);
17   if (p) {
18     assert(p->a >= 42);
19     spin_lock(&p->lock);
20     p = std::kill_dependency(p);
21     p->a++;
22     spin_unlock(&p->lock);
23   }
24 }
\end{verbatim}
}
\caption{Handoff to Locking}
\label{fig:Handoff to Locking}
\end{figure}

Figure~\ref{fig:Handoff to Locking}
shows how RCU protection can hand off to other synchronization
primitives, in this case, locking.
The dependency chain starts at line~16 and continues through line~18
and~19.
However, once line~19 has completed, the code is under the protection
of \co{p->lock}, so line~20 explicitly ends the dependency chain.
The lock then protects the increment on line~21.

It is also possible to hand off protection from RCU to reference counting,
explicit memory barriers, transactional memory, and so on.

Note that the \co{std::kill_dependency()} on line~20 will typically have
no effect on code generation.

\subsection{Evaluation Criteria}
\label{sec:Evaluation Criteria}

\begin{enumerate}
\item	Ease of compilation.
\item	Ease of modification of programs.
\item	Precise specification of dependency chains.
\item	Support for cross-function dependency chains.
\item	Support for cross-compilation-unit dependency chains.
\item	Compatibility with C.
\item	Formal Verification Compatibility.
\end{enumerate}

\section{Marking Proposals}
\label{sec:Marking Proposals}

The following sections present alternative marking proposals.
Whatever proposal is chosen, implementors are encouraged to provide
a means (for example, a command-line argument) to cause the
implementation to act as if markings were placed everywhere they
could reasonably be placed.
This approach permits unmarked programs containing dependency chains
to be handled in a reasonably natural manner.

Note that many of these proposals consist only of short descriptions.
Only proposals having proponents willing to fill them out should be
considered for standardization.

\subsection{Mark Translation Unit}
\label{sec:Mark Translation Unit}

Within the language, translation-unit marking could be accomplished by
a pragma or by a language feature that changed the way pointers are
implemented.
A compiler command-line argument could also be used, but this is of
course outside the standard.
It would be desirable for marked translation units to be able to
be linked with unmarked translation units.

This approach could be useful in cases where only a few of the translation
units contain dependency chains.
However, software-engineering considerations would likely cause many such
projects to mark all the translation units, which would of course result
in the same dependency-chain-tracing complexity as would unmarked
dependency chains.
Any full proposal for this approach should therefore describe how this
issue will be handled.


\subsection{Mark Range of Code}
\label{sec:Mark Range of Code}

Ranges of code could be marked by pragmas, through use of C preprocessor
symbols, or via other ad-hoc means.
However, again, software-engineering considerations would likely cause
many such projects to mark all the translation units, which would of course
result in the same dependency-chain-tracing complexity as would unmarked
dependency chains.
Any full proposal for this approach should therefore describe how this
issue will be handled.


\subsection{Mark Functions}
\label{sec:Mark Functions}

Functions containing dependency chains could be marked with
an attribute (for example, something like
\co{[[function_carries_dependencies]]}) or
a keyword (for example, something like
\co{_FunctionCarriesDependencies}).

Proper use of this approach eliminates issues with dependencies passing
through dependency-unaware code.
However, although there are many software-engineering reasons for
preferring small functions, the fact remains that large functions
are not uncommon in production code.
Large marked functions of course result in similar dependency-chain-tracing
complexity as would unmarked code, so any full proposal for this approach
should describe how this tracing will be handled.


\subsection{Mark Objects}
\label{sec:Mark Objects}

This class of proposals marks the objects that are to carry dependencies.
These objects must be of pointer type.
Note that implementations requiring point-to-point associations between
each \co{memory_order_consume} load and its corresponding dependent
memory references can generate these associations based on the
operations carried out on a given marked object.

\subsubsection{Attribute}
\label{sec:Attribute}

This approach, suggested by Clark Nelson, generalizes the
\co{[[carries_dependency]]} attribute specified in the C++11 standard
so that it applies to objects, including variables, formal parameters,
return values, and class members.
This paper further modifies this proposed attribute so as to also restrict
it to pointer-like objects.

There have been some objections to attributes on the grounds that
attributes are not supposed to change program semantics, but no
consensus as to whether or not this objection is substantive.

The changes to the examples from
Section~\ref{sec:Representative Use Cases}
are similar to those shown in
Section~\ref{sec:Object Modifier}.


\subsubsection{Type Qualifier}
\label{sec:Type Qualifier}

This approach, put forward by Torvald Riegel in response to
Linus Torvalds's spirited criticisms of the current C11 and C++11
wording, introduces a new \co{value_dep_preserving} type qualifier.
Objects marked with this type qualifier carry dependencies.

Again, the changes to the examples from
Section~\ref{sec:Representative Use Cases}
are similar to those shown in
Section~\ref{sec:Object Modifier}.


\subsubsection{Object Modifier}
\label{sec:Object Modifier}

This approach uses a keyword that does not participate in type checking,
for example, a \co{_Carries_dependency} keyword.
This might be treated in a manner similar to a storage class.
It need not necessarily interact with the type system.

\begin{figure}[tbp]
{ \scriptsize
\begin{verbatim}
 1 void thread0(void)
 2 {
 3   struct rcutest *p;
 4
 5   p = (struct rcutest *)malloc(sizeof(*p));
 6   assert(p);
 7   p->a = 42;
 8   assert(p->a != 43);
 9   rcu_assign_pointer(gp, p);
10 }
11
12 void thread1(void)
13 {
14   struct rcutest _Carries_dependency *p;
15
16   p = rcu_dereference(gp);
17   if (p)
18     p->a = 43;
19 }
\end{verbatim}
}
\caption{Object Modifier: Simple Case}
\label{fig:Object Modifier: Simple Case}
\end{figure}

\begin{figure}[tbp]
{ \scriptsize
\begin{verbatim}
 1 void thread0(void)
 2 {
 3   struct rcutest *p;
 4
 5   p = (struct rcutest *)malloc(sizeof(*p));
 6   assert(p);
 7   p->a = 42;
 8   rcu_assign_pointer(gp, p);
 9 }
10
11 void
12 thread1_help(struct rcutest _Carries_dependency *q)
13 {
14   if (q)
15     assert(q->a == 42);
16 }
17
18 void thread1(void)
19 {
20   struct rcutest _Carries_dependency *p;
21
22   p = rcu_dereference(gp);
23   thread1_help(p);
24 }
\end{verbatim}
}
\caption{Object Modifier: In via Function Parameter}
\label{fig:Object Modifier: In via Function Parameter}
\end{figure}

\begin{figure}[tbp]
{ \scriptsize
\begin{verbatim}
 1 void thread0(void)
 2 {
 3   struct rcutest *p;
 4
 5   p = (struct rcutest *)malloc(sizeof(*p));
 6   assert(p);
 7   p->a = 42;
 8   rcu_assign_pointer(gp, p);
 9 }
10
11 struct rcutest _Carries_dependency *thread1_help(void)
12 {
13   return rcu_dereference(gp);
14 }
15
16 void thread1(void)
17 {
18   struct rcutest _Carries_dependency *p;
19
20   p = thread1_help();
21   if (p)
22     assert(p->a == 42);
23 }
\end{verbatim}
}
\caption{Object Modifier: Out via Function Return}
\label{fig:Object Modifier: Out via Function Return}
\end{figure}

\begin{figure}[tbp]
{ \scriptsize
\begin{verbatim}
 1 void thread0(void)
 2 {
 3   struct rcutest *p;
 4
 5   p = (struct rcutest *)malloc(sizeof(*p));
 6   assert(p);
 7   p->a = 42;
 8   rcu_assign_pointer(gp, p);
 9
10   p = (struct rcutest *)malloc(sizeof(*p));
11   assert(p);
12   p->a = 43;
13   rcu_assign_pointer(gsgp, p);
14 }
15
16 struct rcutest _Carries_dependency *
17 thread1_help(struct rcutest _Carries_dependency *p)
18 {
19   if (p)
20     assert(p->a == 42);
21   return rcu_dereference(gsgp);
22 }
23
24 void thread1(void)
25 {
26   struct rcutest _Carries_dependency *p;
27
28   p = rcu_dereference(gp);
29   p = thread1_help(p);
30   if (p)
31     assert(p->a == 43);
32 }
\end{verbatim}
}
\caption{Object Modifier: In and Out, But Different Chains}
\label{fig:Object Modifier: In and Out, But Different Chains}
\end{figure}

\begin{figure}[tbp]
{ \scriptsize
\begin{verbatim}
 1 void thread0(void)
 2 {
 3   struct rcutest *p;
 4
 5   p = (struct rcutest *)malloc(sizeof(*p));
 6   assert(p);
 7   p->a = 42;
 8   rcu_assign_pointer(gp, p);
 9 }
10
11 void
12 thread1_help1(struct rcutest _Carries_dependency *q)
13 {
14   if (q)
15     assert(q->a == 42);
16 }
17
18 void
19 thread1_help2(struct rcutest _Carries_dependency *q)
20 {
21   if (q)
22     assert(q->a != 43);
23 }
24
25 void thread1(void)
26 {
27   struct rcutest _Carries_dependency *p;
28
29   p = rcu_dereference(gp);
30   thread1_help1(p);
31   thread1_help2(p);
32 }
\end{verbatim}
}
\caption{Object Modifier: Chain Fanning Out}
\label{fig:Object Modifier: Chain Fanning Out}
\end{figure}

\begin{figure}[tbp]
{ \scriptsize
\begin{verbatim}
 1 void thread0(void)
 2 {
 3   struct rcutest *p;
 4   struct rcutest1 *p1;
 5
 6   p = (struct rcutest *)malloc(sizeof(*p));
 7   assert(p);
 8   p->a = 42;
 9   rcu_assign_pointer(gp, p);
10
11   p1 = (struct rcutest *)malloc(sizeof(*p1));
12   assert(p1);
13   p1->a = 41;
14   p1->rt.a = 42;
15   rcu_assign_pointer(g1p, p1);
16 }
17
18 void
19 thread1_help(struct rcutest _Carries_dependency *q)
20 {
21   if (q)
22     assert(q->a == 42);
23 }
24
25 void thread1(void)
26 {
27   struct rcutest _Carries_dependency *p;
28
29   p = rcu_dereference(gp);
30   thread1_help(p);
31 }
32
33 void thread2(void)
34 {
35   struct rcutest1 _Carries_dependency *p1;
36
37   p1 = rcu_dereference(g1p);
38   thread1_help(&p1->rt);
39 }
\end{verbatim}
}
\caption{Object Modifier: Chain Fanning In}
\label{fig:Object Modifier: Chain Fanning In}
\end{figure}

\begin{figure}[tbp]
{ \scriptsize
\begin{verbatim}
 1 void thread0(void)
 2 {
 3   struct rcutest *p;
 4   struct rcutest1 *p1;
 5
 6   p = (struct rcutest *)malloc(sizeof(*p));
 7   assert(p);
 8   p->a = 42;
 9   p->b = 43;
10   rcu_assign_pointer(gp, p);
11
12   p1 = (struct rcutest *)malloc(sizeof(*p1));
13   assert(p1);
14   p1->a = 41;
15   p1->rt.a = 42;
16   p1->rt.b = 43;
17   rcu_assign_pointer(g1p, p1);
18 }
19
20 void
21 thread1a_help(struct rcutest _Carries_dependency *q)
22 {
23   assert(q->a == 42);
24 }
25
26 void
27 thread1b_help(struct rcutest _Carries_dependency *q)
28 {
29   assert(q->b == 43);
30 }
31
32 void
33 thread1_help(struct rcutest _Carries_dependency *q)
34 {
35   if (q) {
36     thread1a_help(q);
37     thread1b_help(q);
38   }
39 }
40
41 void thread1(void)
42 {
43   struct rcutest _Carries_dependency *p;
44
45   p = rcu_dereference(gp);
46   thread1_help(p);
47 }
48
49 void thread2(void)
50 {
51   struct rcutest1 _Carries_dependency *p1;
52
53   p1 = rcu_dereference(g1p);
54   thread1_help(&p1->rt);
55 }
\end{verbatim}
}
\caption{Object Modifier: Chain Fanning In and Out}
\label{fig:Object Modifier: Chain Fanning In and Out}
\end{figure}

\begin{figure}[tbp]
{ \scriptsize
\begin{verbatim}
 1 void thread0(void)
 2 {
 3   struct rcutest *p;
 4   struct rcutest1 *p1;
 5
 6   p = (struct rcutest *)malloc(sizeof(*p));
 7   assert(p);
 8   p->a = 42;
 9   p->b = 43;
10   rcu_assign_pointer(gp, p);
11
12   p1 = (struct rcutest *)malloc(sizeof(*p1));
13   assert(p1);
14   p1->a = 41;
15   p1->rt.a = 42;
16   p1->rt.b = 43;
17   rcu_assign_pointer(g1p, p1);
18 }
19
20 #ifdef FOO
21 void
22 thread1a_help(struct rcutest _Carries_dependency *q)
23 {
24   assert(q->a == 42);
25 }
26 #endif
27
28 void
29 thread1b_help(struct rcutest _Carries_dependency *q)
30 {
31   assert(q->b == 43);
32 }
33
34 void
35 thread1_help(struct rcutest _Carries_dependency *q)
36 {
37   if (q) {
38 #ifdef FOO
39     thread1a_help(q);
40 #endif
41     thread1b_help(q);
42   }
43 }
44
45 void thread1(void)
46 {
47   struct rcutest _Carries_dependency *p;
48
49   p = rcu_dereference(gp);
50   thread1_help(p);
51 }
52
53 void thread2(void)
54 {
55   struct rcutest1 _Carries_dependency *p1;
56
57   p1 = rcu_dereference(g1p);
58   thread1_help(&p1->rt);
59 }
\end{verbatim}
}
\caption{Object Modifier: Conditional Compilation of Chain Endpoints}
\label{fig:Object Modifier: Conditional Compilation of Chain Endpoints}
\end{figure}

\begin{figure}[tbp]
{ \scriptsize
\begin{verbatim}
 1 void thread0(void)
 2 {
 3   struct rcutest *p;
 4
 5   p = (struct rcutest *)malloc(sizeof(*p));
 6   assert(p);
 7   p->a = 42;
 8   assert(p->a != 43);
 9   rcu_assign_pointer(gp, p);
10 }
11
12 void thread1(void)
13 {
14   struct rcutest _Carries_dependency *p;
15
16   p = rcu_dereference(gp);
17   if (p) {
18     assert(p->a >= 42);
19     spin_lock(&p->lock);
20     p = std::kill_dependency(p);
21     p->a++;
22     spin_unlock(&p->lock);
23   }
24 }
\end{verbatim}
}
\caption{Object Modifier: Handoff to Locking}
\label{fig:Object Modifier: Handoff to Locking}
\end{figure}

Figures~\ref{fig:Object Modifier: Simple Case}--\ref{fig:Object Modifier: Handoff to Locking}
show how object modifiers can be applied to each of the examples
introduced in Section~\ref{fig:Common Definitions}.
These changes are straightforward markings of local variables, function
parameters, and return-value types.
Object modifiers therefore easily support the use cases in the Linux
kernel.\footnote{
	Give or take a strong distaste for any sort of marking scheme
	on the part of numerous Linux-kernel community members.}


\subsubsection{Template}
\label{sec:Template}

This approach, suggested off-list by JF Bastien, creates a
\co{depending_ptr}\footnote{
	Arbitrarily chosen name with no Google hits.}
template to which a pointer-like type is passed.
This approach allows implementers considerable freedom, as they can
hook into the \co{->} and \co{*} if need be, and also use the C++
\co{delete} keyword to prohibit problematic operations.
Implementations that might nevertheless carry out aggressive
optimizations that might break dependencies even for the non-problematic
operations might need to implement this template class in a manner
similar to the atomics template classes.

This approach would need to be augmented with a non-template solution
for C, for example, the object-modifier approach from
Section~\ref{sec:Object Modifier}.
Implementations that support both C and C++ would presumably relate
Section~\ref{sec:Object Modifier}'s
keyword to the templates in this section in a manner similar to
that used for atomics.

\begin{figure}[tbp]
{ \scriptsize
\begin{verbatim}
 1 template<typename T> class depending_ptr {
 2 public:
 3   depending_ptr(T *v) noexcept;
 4   depending_ptr() noexcept;
 5   bool operator!() noexcept;
 6   class depending_ptr<T>* operator&();
 7   T operator*();
 8   T *operator->();
 9   class depending_ptr<T> operator++();
10   class depending_ptr<T> operator++(int);
11   class depending_ptr<T> operator--();
12   class depending_ptr<T> operator--(int);
13   operator T*();
14   T operator[](long int);
15   bool operator==(T *v) noexcept;
16   bool operator!=(T *v) noexcept;
17   bool operator>(T *v) noexcept;
18   bool operator>=(T *v) noexcept;
19   bool operator<(T *v) noexcept;
20   bool operator<=(T *v) noexcept;
21   class depending_ptr<T> operator+(long idx);
22   class depending_ptr<T> operator+=(long idx);
23   class depending_ptr<T> operator-(long idx);
24   class depending_ptr<T> operator-=(long idx);
25   int operator~() = delete;
26   int operator+() = delete;
27   int operator-() = delete;
28   int operator&(long int) = delete;
29   int operator&=(long int) = delete;
30   int operator%(long int) = delete;
31   int operator%=(long int) = delete;
32   int operator*(long int) = delete;
33   int operator*=(long int) = delete;
34   int operator/(long int) = delete;
35   int operator/=(long int) = delete;
36   int operator<<(long int) = delete;
37   int operator<<=(long int) = delete;
38   int operator>>(long int) = delete;
39   int operator>>=(long int) = delete;
40   int operator^(long int) = delete;
41   int operator^=(long int) = delete;
42   int operator|(long int) = delete;
43   int operator|=(long int) = delete;
44 private:
45   T *dp_rep;
46 };
\end{verbatim}
}
\caption{Template: Declaration}
\label{fig:Template: Declaration}
\end{figure}

\begin{figure}[tbp]
{ \scriptsize
\begin{verbatim}
 1   bool pointer_cmp_eq_dep(void *p, void *q) noexcept;
 2   bool pointer_cmp_ne_dep(void *p, void *q) noexcept;
 3   bool pointer_cmp_gt_dep(void *p, void *q) noexcept;
 4   bool pointer_cmp_ge_dep(void *p, void *q) noexcept;
 5   bool pointer_cmp_lt_dep(void *p, void *q) noexcept;
 6   bool pointer_cmp_le_dep(void *p, void *q) noexcept;
\end{verbatim}
}
\caption{Dependency-Preserving Comparisons}
\label{fig:Dependency-Preserving Comparisons}
\end{figure}

Figure~\ref{fig:Template: Declaration}
shows the resulting template declaration, each member function of which
has a straightforward definition.
Note especially that the relational operators are defined in terms of the
\co{pointer_cmp_eq_dep()},
\co{pointer_cmp_ne_dep()},
\co{pointer_cmp_gt_dep()},
\co{pointer_cmp_ge_dep()},
\co{pointer_cmp_lt_dep()}, and
\co{pointer_cmp_le_dep()}
functions shown in
Figure~\ref{@@@},
so that as long as the first argument to a relational operator is of type
\co{class depending_ptr<T>},
pointers may be safely compared without risk of breaking dependency
chains.\footnote{
	That said, in the prototype implementation,
	these are not intrinsics, but rather separately compiled functions.
	In the absence of link-time optimizations, separate compilation
	preserves dependency chains in most implementations.}
In addition, the operators that cannot be safely applied to
dependency-bearing pointers have been deleted.\footnote{
	The number of pointer-tagging algorithms should motivate
	allowing bitwise operations on dependency-bearing
	pointers, but this should be handled separately.}

\begin{figure}[tbp]
{ \scriptsize
\begin{verbatim}
 1 void thread0(void)
 2 {
 3   struct rcutest *p;
 4
 5   p = (struct rcutest *)malloc(sizeof(*p));
 6   assert(p);
 7   p->a = 42;
 8   assert(p->a != 43);
 9   rcu_store_release(&gp, p);
10 }
11
12 void thread1(void)
13 {
14   depending_ptr<struct rcutest> p;
15
16   p = rcu_consume(&gp);
17   if (p)
18     p->a = 43;
19 }
\end{verbatim}
}
\caption{Template: Simple Case}
\label{fig:Template: Simple Case}
\end{figure}

\begin{figure}[tbp]
{ \scriptsize
\begin{verbatim}
 1 void thread0(void)
 2 {
 3   struct rcutest *p;
 4
 5   p = (struct rcutest *)malloc(sizeof(*p));
 6   assert(p);
 7   p->a = 42;
 8   rcu_store_release(&gp, p);
 9 }
10
11 void
12 thread1_help(depending_ptr<struct rcutest> q)
13 {
14   if (q)
15     assert(q->a == 42);
16 }
17
18 void thread1(void)
19 {
20   depending_ptr<struct rcutest> p;
21
22   p = rcu_consume(&gp);
23   thread1_help(p);
24 }
\end{verbatim}
}
\caption{Template: In via Function Parameter}
\label{fig:Template: In via Function Parameter}
\end{figure}

\begin{figure}[tbp]
{ \scriptsize
\begin{verbatim}
 1 void thread0(void)
 2 {
 3   struct rcutest *p;
 4
 5   p = (struct rcutest *)malloc(sizeof(*p));
 6   assert(p);
 7   p->a = 42;
 8   rcu_store_release(&gp, p);
 9 }
10
11 depending_ptr<struct rcutest> thread1_help(void)
12 {
13   return rcu_consume(&gp);
14 }
15
16 void thread1(void)
17 {
18   depending_ptr<struct rcutest> p;
19
20   p = thread1_help();
21   if (p)
22     assert(p->a == 42);
23 }
\end{verbatim}
}
\caption{Template: Out via Function Return}
\label{fig:Template: Out via Function Return}
\end{figure}

\begin{figure}[tbp]
{ \scriptsize
\begin{verbatim}
 1 void thread0(void)
 2 {
 3   struct rcutest *p;
 4
 5   p = (struct rcutest *)malloc(sizeof(*p));
 6   assert(p);
 7   p->a = 42;
 8   rcu_store_release(&gp, p);
 9
10   p = (struct rcutest *)malloc(sizeof(*p));
11   assert(p);
12   p->a = 43;
13   rcu_store_release(&gsgp, p);
14 }
15
16 depending_ptr<struct rcutest> 
17 thread1_help(depending_ptr<struct rcutest> p)
18 {
19   if (p)
20     assert(p->a == 42);
21   return rcu_consume(&gsgp);
22 }
23
24 void thread1(void)
25 {
26   depending_ptr<struct rcutest> p;
27
28   p = rcu_consume(&gp);
29   p = thread1_help(p);
30   if (p)
31     assert(p->a == 43);
32 }
\end{verbatim}
}
\caption{Template: In and Out, But Different Chains}
\label{fig:Template: In and Out, But Different Chains}
\end{figure}

\begin{figure}[tbp]
{ \scriptsize
\begin{verbatim}
 1 void thread0(void)
 2 {
 3   struct rcutest *p;
 4
 5   p = (struct rcutest *)malloc(sizeof(*p));
 6   assert(p);
 7   p->a = 42;
 8   rcu_store_release(&gp, p);
 9 }
10
11 void
12 thread1_help1(depending_ptr<struct rcutest> q)
13 {
14   if (q)
15     assert(q->a == 42);
16 }
17
18 void
19 thread1_help2(depending_ptr<struct rcutest> q)
20 {
21   if (q)
22     assert(q->a != 43);
23 }
24
25 void thread1(void)
26 {
27   depending_ptr<struct rcutest> p;
28
29   p = rcu_consume(&gp);
30   thread1_help1(p);
31   thread1_help2(p);
32 }
\end{verbatim}
}
\caption{Template: Chain Fanning Out}
\label{fig:Template: Chain Fanning Out}
\end{figure}

\begin{figure}[tbp]
{ \scriptsize
\begin{verbatim}
 1 void thread0(void)
 2 {
 3   struct rcutest *p;
 4   struct rcutest1 *p1;
 5
 6   p = (struct rcutest *)malloc(sizeof(*p));
 7   assert(p);
 8   p->a = 42;
 9   rcu_store_release(&gp, p);
10
11   p1 = (struct rcutest *)malloc(sizeof(*p1));
12   assert(p1);
13   p1->a = 41;
14   p1->rt.a = 42;
15   rcu_store_release(&g1p, p1);
16 }
17
18 void
19 thread1_help(depending_ptr<struct rcutest> q)
20 {
21   if (q)
22     assert(q->a == 42);
23 }
24
25 void thread1(void)
26 {
27   depending_ptr<struct rcutest> p;
28
29   p = rcu_consume(&gp);
30   thread1_help(p);
31 }
32
33 void thread2(void)
34 {
35   depending_ptr<struct rcutest1> p1;
36
37   p1 = rcu_consume(&g1p);
38   thread1_help(&p1->rt);
39 }
\end{verbatim}
}
\caption{Template: Chain Fanning In}
\label{fig:Template: Chain Fanning In}
\end{figure}

\begin{figure}[tbp]
{ \scriptsize
\begin{verbatim}
 1 void thread0(void)
 2 {
 3   struct rcutest *p;
 4   struct rcutest1 *p1;
 5
 6   p = (struct rcutest *)malloc(sizeof(*p));
 7   assert(p);
 8   p->a = 42;
 9   p->b = 43;
10   rcu_store_release(&gp, p);
11
12   p1 = (struct rcutest *)malloc(sizeof(*p1));
13   assert(p1);
14   p1->a = 41;
15   p1->rt.a = 42;
16   p1->rt.b = 43;
17   rcu_store_release(&g1p, p1);
18 }
19
20 void
21 thread1a_help(depending_ptr<struct rcutest> q)
22 {
23   assert(q->a == 42);
24 }
25
26 void
27 thread1b_help(depending_ptr<struct rcutest> q)
28 {
29   assert(q->b == 43);
30 }
31
32 void
33 thread1_help(depending_ptr<struct rcutest> q)
34 {
35   if (q) {
36     thread1a_help(q);
37     thread1b_help(q);
38   }
39 }
40
41 void thread1(void)
42 {
43   depending_ptr<struct rcutest> p;
44
45   p = rcu_consume(&gp);
46   thread1_help(p);
47 }
48
49 void thread2(void)
50 {
51   depending_ptr<struct rcutest1> p1;
52
53   p1 = rcu_consume(&g1p);
54   thread1_help(&p1->rt);
55 }
\end{verbatim}
}
\caption{Template: Chain Fanning In and Out}
\label{fig:Template: Chain Fanning In and Out}
\end{figure}

\begin{figure}[tbp]
{ \scriptsize
\begin{verbatim}
 1 void thread0(void)
 2 {
 3   struct rcutest *p;
 4   struct rcutest1 *p1;
 5
 6   p = (struct rcutest *)malloc(sizeof(*p));
 7   assert(p);
 8   p->a = 42;
 9   p->b = 43;
10   rcu_store_release(&gp, p);
11
12   p1 = (struct rcutest *)malloc(sizeof(*p1));
13   assert(p1);
14   p1->a = 41;
15   p1->rt.a = 42;
16   p1->rt.b = 43;
17   rcu_store_release(&g1p, p1);
18 }
19
20 #ifdef FOO
21 void
22 thread1a_help(depending_ptr<struct rcutest> q)
23 {
24   assert(q->a == 42);
25 }
26 #endif
27
28 void
29 thread1b_help(depending_ptr<struct rcutest> q)
30 {
31   assert(q->b == 43);
32 }
33
34 void
35 thread1_help(depending_ptr<struct rcutest> q)
36 {
37   if (q) {
38 #ifdef FOO
39     thread1a_help(q);
40 #endif
41     thread1b_help(q);
42   }
43 }
44
45 void thread1(void)
46 {
47   depending_ptr<struct rcutest> p;
48
49   p = rcu_consume(&gp);
50   thread1_help(p);
51 }
52
53 void thread2(void)
54 {
55   depending_ptr<struct rcutest1> p1;
56
57   p1 = rcu_consume(&g1p);
58   thread1_help(&p1->rt);
59 }
\end{verbatim}
}
\caption{Template: Conditional Compilation of Chain Endpoints}
\label{fig:Template: Conditional Compilation of Chain Endpoints}
\end{figure}

\begin{figure}[tbp]
{ \scriptsize
\begin{verbatim}
 1 void thread0(void)
 2 {
 3   struct rcutest *p;
 4
 5   p = (struct rcutest *)malloc(sizeof(*p));
 6   assert(p);
 7   p->a = 42;
 8   assert(p->a != 43);
 9   rcu_store_release(&gp, p);
10 }
11
12 void thread1(void)
13 {
14   depending_ptr<struct rcutest> p;
15
16   p = rcu_consume(&gp);
17   if (p) {
18     assert(p->a >= 42);
19     spin_lock(&p->lock);
20     p = std::kill_dependency(p);
21     p->a++;
22     spin_unlock(&p->lock);
23   }
24 }
\end{verbatim}
}
\caption{Template: Handoff to Locking}
\label{fig:Template: Handoff to Locking}
\end{figure}

Figures~\ref{fig:Template: Simple Case}--\ref{fig:Template: Handoff to Locking}
show how templates can be applied to each of the examples
introduced in Section~\ref{sec:Representative Use Cases}.
As with the object-modifier approach in
Section~\ref{sec:Object Modifier},
these changes are straightforward markings of local variables, function
parameters, and return-value types.

Full source code for a prototype implementation
(and for this paper) may be downloaded from
{\small \co{https://github.com/paulmckrcu/2016markconsume.git}}.


\subsection{Mark Root/Leaf Pairs}
\label{sec:Mark Root/Leaf Pairs}

These approaches create point-to-point associations between
\co{memory_order_consume} loads and the memory references that depend
on them.
Function calls can be handled by using the arguments of the function
call and the function parameters as intermediate points in the
association.
Function returns can be handled by using the function return
declaration and the function return value.

However, these point-to-point associations are required to gracefully handle
bushy dependency trees, dependency trees that fan both in and out,
and conditional compilation.
Any scheme that relies on directly referencing a specific location
in the source code will fall afoul of these requirements.

One approach is to use a unique identifier for each dependency tree,
and associate each relevant point in the code with the corresponding
identifiers.


\section{Evaluation}
\label{sec:Evaluation}

\begin{table}
\small
\centering
\begin{tabular}{l|c|c|c|c|c|c|c}
Mark:
	~ ~ ~ ~ ~ ~ ~ ~ ~
	& \begin{picture}(6,185)(0,0)
		\rotatebox{90}{Ease of Compilation}
	  \end{picture}
	& \begin{picture}(6,185)(0,0)
		\rotatebox{90}{Ease of Modification}
	  \end{picture}
	& \begin{picture}(6,185)(0,0)
		\rotatebox{90}{Precise Dependency Chains}
	  \end{picture}
	& \begin{picture}(6,185)(0,0)
		\rotatebox{90}{Cross-Function Dependency Chains}
	  \end{picture}
	& \begin{picture}(6,185)(0,0)
		\rotatebox{90}{Cross-Compilation-Unit Dependency Chains}
	  \end{picture}
	& \begin{picture}(6,185)(0,0)
		\rotatebox{90}{C Compatibility}
	  \end{picture}
	& \begin{picture}(6,185)(0,0)
		\rotatebox{90}{Formal Verification}
	  \end{picture}
	\\
	\hline
%	 EC  EM  PD    CF  CC  CC  FV
\hline
Translation Unit
	& T &   & N   &   & m &   &  N \\
\hline
Range of Code
	& t &   & N   & m & m &   &  N \\
\hline
Functions
	& t &   & N   & m & m &   &  N \\
\hline
Objects
	&   &   &     &   &   &   &    \\
\hline
~~~~Attribute
	&   & o &     & t & t & a &    \\
\hline
~~~~Type Qualifier
	&   & o &     &   &   &   &  N \\
\hline
~~~~Modifier
	&   & o &     & t & t &   &    \\
\hline
~~~~Template
	&   & o &     &   &   & N &    \\
\hline
~~~~Template+Modifier
	&   & o &     &   &   &   &    \\
\hline
Root/Leaf
	&   & A &     & ? & ? & ? &  ? \\
\hline
Nothing
	&   &   & N   &   &   &   &  N \\
\end{tabular}
\caption{Dependency-Chain Marking Evaluation}
\label{tab:Dependency-Chain Marking Evaluation}
\end{table}

Table~\ref{tab:Dependency-Chain Marking Evaluation}
provides a rough comparison between the various marking methods, and
also includes the unmarked option for comparison purposes.

For ease of compilation, the cells corresponding to methods that
explicitly mark dependency chains or that don't require marking at all
are left blank.
Those that require tracing dependency chains throughout the full
translation unit are marked ``T'' and those that limit the code in
which tracing is required are marked ``t''.

For ease of modification, the cells corresponding to methods that
either require no marking or that mark large-scale entities are
left blank.
Those that require marking the definitions of objects that carry
dependencies are marked ``o'', and those require marking of individual
accesses are marked ``A''.

Cells corresonding to those methods that precisely mark dependency chains are 
left blank, otherwise, they are marked ``N''.

For cross-function dependency chains, those methods that either support
cross-function marking or that do not require such marking are left blank.
Those that require manual consistency checks are marked ``m'', those
that are amenable to consistency-check tooling are marked ``t'', and
those that are not fleshed out sufficiently to tell are marked ``?''.
These same markings are used for cross-compilation-unit dependency
chains.

Cells corresponding to those methods supporting C compatibility are left
blank.
Those that would support C compatibility if C were to provide attributes
are marked ``a''.
Those that do not support C compatibility (at least not unless combined
with some other method) are marked ``N'',
and those that are not fleshed out sufficiently to tell are marked ``?''.

Cells corresponding to methods believed to support formal verification
are left blank, those that are believed not to support formal verification
are marked ``N'',
and those that are not fleshed out sufficiently to tell are marked ``?''.
Note that the object type qualifier could in theory support formal
verification, but the specific proposal rules this out by requiring that
the compiler treat \co{memory_order_consume} loads as potentially
returning any value from the type.

Following the lead of C11 and C++11 atomics, the ``Template+Modifier''
row covers the combination of
marking objects with template classes (for C++) and with an
typed object modifier (for C), a combination that appears to be quite
attractive.
This should be further combined with a totally unmarked option for
use by standalone projects such as the Linux kernel.

The best possible method would have a row of all blank cells.

\section{Summary}
\label{sec:Summary}

This paper reviewed the 2016 discussions of \co{memory_order_consume}
that took place at the Jacksonville meeting, presented several representative
use cases, listed evaluation criteria, presented a number of
marking proposals (two in depth), and provided a comparative
evaluation.

A combination of typed object modifier (for C compatibility) and
a template class (for C++), similar to the approach used by atomics.
For standalone applications, these should be further combined with an
unmarked option, where the implementation assumes that everything that
could legally marked is so marked.

%\input{acknowledgments}
%\input{legal}

%\bibliographystyle{abbrv}
%\softraggedright

\bibliographystyle{acm}
\bibliography{bib/RCU,bib/WFS,bib/hw,bib/os,bib/parallelsys,bib/patterns,bib/perfmeas,bib/refs,bib/syncrefs,bib/search,bib/swtools,bib/realtime,bib/TM,bib/standards,bib/maze}

% \section*{Change Log}
% \label{sec:Change Log}

% This paper first appeared as {\bf N4026} in May of 2014.
% Revisions to this document are as follows:

% \begin{itemize}
% \item	Mark the paper as a revision of N4036.
% 	(July 17, 2014.)
% \end{itemize}

% At this point, the paper was published as {\bf P0098R1}.

\end{document}

%%%%%%%%%%%%%%%%%%%%%%%%%%%%%%%%%%%%%%%%%%%%%%%%%%%%%%%%%%%%%%%%%%%%%%%%%%%%%%
% for Ispell:
% LocalWords:  workingdraft BCM ednote SubSections xfig SubSection

