These approaches create point-to-point associations between
\co{memory_order_consume} loads and the memory references that depend
on them.
Function calls can be handled by using the arguments of the function
call and the function parameters as intermediate points in the
association.
Function returns can be handled by using the function return
declaration and the function return value.

However, these point-to-point associations are required to gracefully handle
bushy dependency trees, dependency trees that fan both in and out,
and conditional compilation.
Any scheme that relies on directly referencing a specific location
in the source code will fall afoul of these requirements.

One approach is to use a unique identifier for each dependency tree,
and associate each relevant point in the code with the corresponding
identifiers.

Note that the root-leaf information could in theory be extracted by
the compiler based on object markings
(see Section~\ref{sec:Mark Objects}).
